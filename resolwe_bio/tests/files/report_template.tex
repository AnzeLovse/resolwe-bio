\documentclass[11pt]{article}

% Graphics, plotting, images:
\usepackage{graphicx}

% Tweaking text borders:
\usepackage[top=3cm, bottom=3cm, left=1.7cm, right=2cm]{geometry}

% Making hyperlinks:
\usepackage{hyperref}

% Making headers/footers:
\usepackage{fancyhdr}
\pagestyle{fancy}
\lhead{\includegraphics[width=35mm]{{#LOGO#}}}

\usepackage{caption}

% Make only a subset of pages in landscape mode
\usepackage{pdflscape}

% tables spanning multiple pages
\usepackage{longtable}

\usepackage{array}
\newcolumntype{L}{>{\centering\arraybackslash}m{2.3cm}}

\begin{document}

\section*{{#SAMPLE_NAME#}}

\noindent
{\Large {#PANEL#}} \\

\noindent
{\large Sample analysis report} \\
\noindent \today

\section{QC information}

\subsection*{Amplicon performance}

Total reads: \textbf{{#TOTAL_READS#}} \\
Aligned reads: \textbf{{#ALIGNED_READS#}} \\
Bases on target (aligned): \textbf{{#BASES_ALIGNED#}} \% \\
Bases on target (total): \textbf{{#BASES_TARGET#}} \% \\
Mean coverage: \textbf{{#COV_MEAN#}} \\
Mean coverage (20 \% of Mean) : \textbf{{#COV_20#}} \\
Coverage uniformity (\% bases covered above 20 \% of Mean): \textbf{{#COV_UNI#}} \%  \\


\subsection*{Per amplicon coverage}
Number of amplicons: \textbf{{{#ALL_AMPLICONS#}}} \\
Number of amplicons with 100 \% coverage: \textbf{{#COVERED_AMPLICONS#} ({#PCT_COV#} \%)} \\

{#BAD_AMPLICON_TABLE#}

\newpage

\subsection*{Per amplicon average coverage}
\begin{figure}[h]
    \centering
    \includegraphics[width=0.99\textwidth]{{#IMAGE2#}}
    \caption{Average coverage (average number of sequencing reads
    covering amplicon bases) is plotted for
    each of the tested amplicons. Coverage at the 5 \%, 10 \%, 20 \% and 5x of the mean sample coverage
    is marked with the horizontal lines.}
\end{figure}

\newpage

\begin{landscape}
    \section{Annotated variants}
    \small
    {#VCF_TABLES#}
\end{landscape}


\end{document}
